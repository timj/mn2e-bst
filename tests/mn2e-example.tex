% mn2e-example.tex
%
% This is a skeleton MN-format article which displays the generated
% test bibliography (mn2e-test.bbl) as it would appear in an article.
% To see this, try:
%
%     make mn2e-example.pdf
%
% Note that the mn2e-test.aux file is written by hand; the
% mn2e-example.aux file generated from this file is ignored.

\documentclass[useAMS,usenatbib]{mn2e}

\title[Illustrations of Citations]{Illustrations of Citations}
\author[Albert One and Benedict Two]{Albert One$^{1}$\thanks{E-mail:
ao@example.ac.uk (AO)} and Benedict Two$^{2}$\\
$^{1}$A Place, Somewhere\\
$^{2}$Elsewhere, On Earth}

% mn2e class file should either import the hyperref package, or else
% provide suitable dummys for \href and \url
\usepackage{hyperref}

\makeatletter
% The following three macros -- \doi, \eprint and \mniiiauthor --
% should be migrated to mn2e.cls.

% \doi{10.foo} formats the DOI in the argument, and provides a link to dx.doi.org.
% \doi[text]{10.foo} formats the DOI 10.foo, but provides 'text' as the link.
% The DOI can contain {\$&#^_%~} (though there's not necessarily a
% guarantee that these will still work as URL characters within the PDF)
\def\doi{\begingroup
  % The following isn't just \dospecials, because that includes \ , \{, and \}
  \let\do\@makeother \do\\\do\$\do\&\do\#\do\^\do\_\do\%\do\~
  \@ifnextchar[%]
    {\@doi}
    {\@doi[]}}
\def\@doi[#1]#2{%
  \def\@tempa{#1}%
  \ifx\@tempa\@empty
    \href{http://dx.doi.org/#2}{doi:#2}%
  \else
    \href{http://dx.doi.org/#2}{#1}%
  \fi
  \endgroup
}

% \eprint{defaultArchivePrefix}{id} expands to a link to the given ID
% at a suitable archive.  The 'id' can be either a bare ID (such as
% yymm.1234) for arXiv, or can include an archive prefix.  If there is
% no prefix in the 'id', then 'defaultArchivePrefix' supplies a default.
%
% Thus
%   \eprint{}{arXiv:yymm.1234} -> \href{http://arxiv.org/abs/yymm.1234}{arXiv:yymm.1234}
%   \eprint{}{yymm.1234} -> same as \eprint{}{arXiv:yymm.1234}
%   \eprint{arXiv}{arXiv:yymm.1234} -> same
%   \eprint{dblp}{1234} -> \href{http://dblp.uni-trier.de/rec/bibtex/1234.xml}{dblp:1234}
%   \eprint{dblp}{arXiv:yymm.1234} -> same as \eprint{}{arXiv:yymm.1234}
%   \eprint{}{wibble:1234} -> wibble:1234 (doesn't match anything)
\def\eprint#1#2{%
  \@eprint#1:#2::\@nil}
\def\@eprint@arXiv#1{\href{http://arxiv.org/abs/#1}{{\tt arXiv:#1}}}
\def\@eprint@dblp#1{\href{http://dblp.uni-trier.de/rec/bibtex/#1.xml}{dblp:#1}}
\def\@eprint#1:#2:#3:#4\@nil{%
  \def\@tempa{#1}%
  \def\@tempb{#2}%
  \def\@tempc{#3}%
  \ifx\@tempc\@empty
    \let\@tempc\@tempb
    \let\@tempb\@tempa
  \fi
  \ifx\@tempb\@empty
    % default to arXiv
    \def\@tempb{arXiv}%
  \fi
  % If \@tempb is a 'recognised' prefix, then call it, otherwise, just
  % print prefix:id and be done with it.  A prefix is 'recognised' if
  % there's a macro \@eprint@<prefix>.
  \@ifundefined{@eprint@\@tempb}
    {\@tempb:\@tempc}
    {% call macro '@eprint@\@tempb' on the argument \@tempc
      \expandafter\expandafter\csname @eprint@\@tempb\endcsname\expandafter{\@tempc}}%
}

% The following implements the three-author-hack described in
% mn2e.bst.  This should be moved to mn2e.cls at some point.
%
% This consumes a command for each such author.  It's surely possible
% to avoid this (with some constructions involving {\\#1}; see
% Appendix D cleverness), but that would verge on the arcane, and not
% be really worth it.
\def\mniiiauthor#1#2#3{%
  \@ifundefined{mniiiauth@#1}
    {\global\expandafter\let\csname mniiiauth@#1\endcsname\null #2}
    {#3}}

% ...end of prospective mn2e.cls material
\makeatother

\begin{document}

\date{Whenever}

\pagerange{\pageref{firstpage}--\pageref{lastpage}} \pubyear{2002}

\maketitle

\label{firstpage}

\begin{abstract}
Stuff.
\end{abstract}

\begin{keywords}
keywords
\end{keywords}

% Tests of \eprint
%\eprint{}{arXiv:yymm.1234} %-> \href{http://arxiv.org/abs/yymm.1234}{arXiv:yymm.1234}
%\eprint{}{yymm.1234} %-> same as \eprint{}{arXiv:yymm.1234}
%\eprint{arXiv}{arXiv:yymm.1234} %-> same
%\eprint{dblp}{1234} %-> \href{http://dblp.uni-trier.de/rec/bibtex/1234.xml}{dblp:1234}
%\eprint{dblp}{arXiv:yymm.1234} %-> same as \eprint{}{arXiv:yymm.1234}
%\eprint{}{wibble:1234} %-> wibble:1234 (doesn't match anything)


\section{Introduction}

There are numerous citations here.

In particular, there is
`one'~\citep{one},
`oneplus'~\citep{oneplus},
`onereprised'~\citep{onereprised},
`two'~\citep{two},
`twoplus'~\citep{twoplus},
`twobis'~\citep{twobis},
`twobook'~\citep{twobook},
`twomisc~\citep{twomisc},
`tworeprised'~\citep{tworeprised},
`three'~\citep{three},
`threeplus'~\citep{threeplus},
`threereprised'~\citep{threereprised},
`four'~\citep{four},
`fourplus'~\citep{fourplus},
`seven'~\citep{seven},
`sevenplus'~\citep{sevenplus},
`eight'~\citep{eight},
`eightplus'~\citep{eightplus},
`nine'~\citep{nine},
`nineplus'~\citep{nineplus},
`ten'~\citep{ten},
`tenplus'~\citep{tenplus} and
`tenbis'~\citep{tenbis}.

And secondly, there is
`one'~\citep{one},
`oneplus'~\citep{oneplus},
`onereprised'~\citep{onereprised},
`two'~\citep{two},
`twoplus'~\citep{twoplus},
`twobis'~\citep{twobis},
`twobook'~\citep{twobook},
`twomisc~\citep{twomisc},
`tworeprised'~\citep{tworeprised},
`three'~\citep{three},
`threeplus'~\citep{threeplus},
`threereprised'~\citep{threereprised},
`four'~\citep{four},
`fourplus'~\citep{fourplus},
`seven'~\citep{seven},
`sevenplus'~\citep{sevenplus},
`eight'~\citep{eight},
`eightplus'~\citep{eightplus},
`nine'~\citep{nine},
`nineplus'~\citep{nineplus},
`ten'~\citep{ten},
`tenplus'~\citep{tenplus} and
`tenbis'~\citep{tenbis}.



% ../mn2e.bst can generate a \logsortkey in the output, for
% debugging.  That's generally commented out, but if you need to
% uncomment that, then uncomment this, too.
%\def\logsortkey#1{{[\tiny #1]}}
\input{mn2e-test.bbl}

\bsp

\label{lastpage}

\end{document}
